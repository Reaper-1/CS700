\documentclass[12pt]{article}

% Packages
\usepackage{amsmath} 
\usepackage{graphicx}
\usepackage{float}
\usepackage{hyperref}
\usepackage{listings}
\usepackage{geometry}
\geometry{margin=1in}

\begin{document}

\title{\textbf{CS-700 Algorithms and Complexity
Assignment: Comparison of Sorting Algorithms}}
\author{Abhijith C}
\date{02/09/2024}

\maketitle

\tableofcontents

\newpage

\section{Introduction}
    The objective of the assignment is to analyze the theoretical and actual performance of different sorting algorithms and compare them. The sorting algorithms studied in this assignment are :
    
    \begin{itemize}
        \item Bubble Sort
        \item Insertion Sort
        \item Merge Sort
        \item Quick Sort
        \item Heap Sort
        \item Radix Sort
    \end{itemize}
    
    For Quick Sort, the different strategies used for selecting the pivot are:
    
    \begin{itemize}
        \item Pivot Choice 1: The first element in the array.
        \item Pivot Choice 2: A random element in the array.
        \item Pivot Choice 3: The median of the array's first, middle, and last elements.
    \end{itemize}
    
    The best version of Quick Sort is considered for comparison with other algorithms.

\section{Theoretical Overview}


\begin{itemize}
    \item \textbf{Bubble Sort:} \(O(n^2)\) in the worst and average cases, \(O(n)\) in the best case.
    \item \textbf{Insertion Sort:} \(O(n^2)\) in the worst and average cases, \(O(n)\) in the best case.
    \item \textbf{Merge Sort:} \(O(n \log n)\) in all cases.
    \item \textbf{Quick Sort:} \(O(n^2)\) in the worst case, \(O(n \log n)\) on average and in the best case.
    \item \textbf{Heap Sort:} \(O(n \log n)\) in all cases.
    \item \textbf{Radix Sort:} \(O(n \cdot k)\), where \(k\) is the number of digits in the largest number.
\end{itemize}

\section{Experimental Setup}

    \subsection{Machine Used}
    The specifications of the machine used for performing the experiment are :
        \begin{itemize}
            \item Hardware Model: HP HP EliteDesk 800 G8 Tower PC
            \item Memory: 16.0 GiB
            \item Processor : 11th Gen Intel® Core™ i5-11500 @ 2.70GHz × 12 
            \item Graphics: Mesa Intel® Graphics (RKL GT1)
            \item Disk capacity: 1.0 TB
         
            \item OS Name: Ubuntu 22.04 LTS
            \item OS Type: 64-bit
            \item Gnome Version: 42.0
            \item Windowing System: Wayland
        \end{itemize}

    \subsection{Timing Mechanism}
    The timing mechanism used to record the execution times of the programs is the clock() function in cpp. The clock() function returns the CPU time used by the program.

    \subsection{Reported Times}
    The experiment was repeated three times, and the reported times represent the average of the values obtained from these repetitions
    
    \subsection{Inputs}
    The different sorting algorithms are tested against the same input to ensure a fair comparison of performance. Three types of inputs used for testing are:
    
    \begin{itemize}
        \item Random inputs
        \item Inputs in increasing order
        \item Inputs in decreasing order. 
    \end{itemize}

    The inputs were generated with sizes ranging from 10,000 to 100,000, increasing in increments of 10,000 and stored in a text file, which was then given as input to the program.

\subsection{Quick Sort Variants}

\subsection{Overall Performance Comparison}


\section{Conclusion}


\section{References}

\end{document}