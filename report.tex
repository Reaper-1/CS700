\documentclass[12pt]{article}

% Packages
\usepackage{amsmath}   % For advanced math typesetting
\usepackage{graphicx}  % For including images
\usepackage{float}     % For controlling figure and table placement
\usepackage{hyperref}  % For hyperlinks in the document
\usepackage{listings}  % For code listings
\usepackage{geometry}  % For customizing page dimensions
\geometry{margin=1in}  % Setting the margins

% Title and Author
\title{Assignment: Comparative Analysis of Sorting Algorithms}
\author{[Your Name]}
\date{[Date]}

\begin{document}

\maketitle

\tableofcontents

\newpage

\section{Introduction}


\section{Theoretical Overview}


\begin{itemize}
    \item \textbf{Bubble Sort:} \(O(n^2)\) in the worst and average cases, \(O(n)\) in the best case.
    \item \textbf{Insertion Sort:} \(O(n^2)\) in the worst and average cases, \(O(n)\) in the best case.
    \item \textbf{Merge Sort:} \(O(n \log n)\) in all cases.
    \item \textbf{Quick Sort:} \(O(n^2)\) in the worst case, \(O(n \log n)\) on average and in the best case.
    \item \textbf{Heap Sort:} \(O(n \log n)\) in all cases.
    \item \textbf{Radix Sort:} \(O(n \cdot k)\), where \(k\) is the number of digits in the largest number.
\end{itemize}

\section{Experimental Analysis}


%\begin{figure}[H]
%\centering
%\includegraphics[width=0.8\textwidth]{performance_graph.png}
%\caption{Comparison of Sorting Algorithms' Performance}
%\label{fig:performance}
%\end{figure}

\subsection{Quick Sort Variants}
Analyze the performance of the three Quick Sort variants:


\subsection{Overall Performance Comparison}


\section{Conclusion}


\section{References}

\end{document}
